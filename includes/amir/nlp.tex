\section{Natural Language Processing}
\label{chap:Natural Language Processing}

  %%%%%%%%%%%%%%%%%%%%%%%%%%%%
  % SUBSECTION               %
  %%%%%%%%%%%%%%%%%%%%%%%%%%%%
\subsection{Introduction}
Natural language processing (NLP) is an area of computer science and AI concerned with the interactions between computers and human (natural) languages, in particular how to program computers to process and analyze large amounts of natural language data. NLP draws from many disciplines, including computer science and computational linguistics, in its pursuit to fill the gap between human communication and computer understanding.\cite{web011}\\
While natural language processing isn’t a new science, the technology is rapidly advancing thanks to an increased interest in human-to-machine communications, plus an availability of big data, powerful computing and enhanced algorithms. The complete interaction was made possible by NLP, along with other AI elements such as machine learning and deep learning.\\
\subsection{Challenges}
NLP is characterized as a hard problem in computer science. To understand human language is to understand not only the words, but the concepts and how they’re linked together to create meaning. Despite language being one of the easiest things for humans to learn, the ambiguity and linguistic variation of language is what makes natural language processing a difficult problem for computers to master.\\ NLP is important because it helps resolve ambiguity in language and adds useful numeric structure to the data.\\
Challenges in natural language processing frequently involve speech recognition, natural language understanding, and natural language generation.\\
\subsection{Major Evaluations and Tasks}
NLP is used to analyze text, allowing machines to understand how human’s speak.
By utilizing NLP, developers can organize and structure knowledge to perform tasks.\\
Speech Tasks: Speech recognition, Speech recognition, Text-to-speech.\\
Discourse Tasks: Automatic summarization, Coreference resolution, Discourse analysis.\\
Syntax Tasks: Morphological segmentation, Part-of-speech tagging, Parsing, Sentence breaking, Word segmentation, Terminology extraction.\\
Semantics Tasks: Lexical semantics, Machine translation, Named entity recognition (NER), Natural language generation, Natural language understanding, Optical character recognition (OCR), Recognizing Textual entailment, Relationship extraction, multimodal sentiment analysis, Topic segmentation and recognition, Word sense disambiguation, and automated Question answering (QA).
\subsection{Large Volumes of Textual Data}
NLP makes it possible for computers to read text, hear speech, interpret it, measure sentiment and determine which parts are important. Today’s machines can analyze more language-based data than humans.\\NLP algorithms are typically based on machine learning algorithms. Instead of hand-coding large sets of rules, NLP can rely on machine learning to automatically learn these rules by analyzing a set of examples (i.e. a large corpus, like a book, down to a collection of sentences), and making a statical inference. In general, the more data analyzed, the more accurate the model will be.
\subsection{NLP and IR}
Information retrieval addresses the problem of finding those documents whose content matches a user's request from among a large collection of documents.
NLP techniques are often used for facilitating descriptions of document content and for presenting the user's query.\\Attempts to improve retrieval performance through more sophisticated linguistic processing have been largely unsuccessful. Indeed such processing can degrade retrieval effectiveness.\\
In other words, a textual information retrieval system carries out the following tasks in response to a user's query.\\
\indent$\bullet$\hspace{5pt}Indexing the collection of documents: in this phase, NLP techniques are applied to generate an index containing document descriptions. Normally each document is described through a set of terms that best represents its content.\\
\indent$\bullet$\hspace{5pt}When a user formulates a query, the system analyses it, and if necessary, transforms it with the hope of representing the user's information needs in the same way as the document content is represented.\\
\indent$\bullet$\hspace{5pt}The system compares the description of each document with that of the query, and presents the user with those documents whose descriptions are closest to the query description.\\
\indent$\bullet$\hspace{5pt}The results are usually listed in order of relevancy, that is, by the level of similarity between the document and query descriptions.
\\\\
Statistical processing of natural language represents the classical model of information retrieval systems, and is characterized from each document's set of key words, known as the terms index.\\
This is a very simple focus based on the "bag of words." In this approach, all words in a document are treated as its index terms. Moreover, each term is assigned a weight in function of its importance, usually determined by its appearance frequency within the document. This way the word's order, structure, meaning, etc, are not taken into consideration.\\
\subsection{Techniques and Apps}
NLP includes many different techniques for interpreting human language, ranging from statistical and machine learning methods to rules-based and algorithmic approaches.
Basic NLP tasks include tokenization and parsing, lemmatization/stemming, part-of-speech tagging, language detection and identification of semantic relationships\\
\indent$\bullet$\hspace{5pt}Create a chat bot using Parsey McParseface, a language parsing deep learning model made by Google that uses Point-of-Speech tagging.\\
\indent$\bullet$\hspace{5pt}Summarize blocks of text using Summarizer to extract the most important and central ideas while ignoring irrelevant information.\\
\indent$\bullet$\hspace{5pt}Identify the type of entity extracted, such as it being a person, place, or organization using Named Entity Recognition.\\
\indent$\bullet$\hspace{5pt}Reduce words to their root, or stem, using PorterStemmer, or break up text into tokens using Tokenizer.\\
\indent$\bullet$\hspace{5pt}Use Sentiment Analysis to identify the sentiment of a string of text, from very negative to neutral to very positive.\\
\indent$\bullet$\hspace{5pt}Machine translation. Automatic translation of text or speech from one language to another.\\
\indent$\bullet$\hspace{5pt}Speech-to-text and text-to-speech conversion. Transforming voice commands into written text, and vice versa.\\
In all these cases, the overarching goal is to take raw language input and use linguistics and algorithms to transform or enrich the text in such a way that it delivers greater value.\\
Text analytics is used to explore textual content and derive new variables from raw text that may be visualized, filtered, or used as inputs to predictive models or other statistical methods\\
There are many common and practical applications of NLP in our everyday lives. Beyond conversing with virtual assistants like Alexa or Siri, here are a few more examples:\\
\indent$\bullet$\hspace{5pt}Bayesian spam filtering, a statistical NLP technique that compares the words in spam to valid emails to identify junk mail.\\
\indent$\bullet$\hspace{5pt}Identify patterns and clues in emails or written reports to help detect and solve crimes.\\
\indent$\bullet$\hspace{5pt}Social media analytics. Track awareness and sentiment about specific topics and identify key influencers.\\\\
categorization.
A subfield of NLP called natural language understanding (NLU) has begun to rise in popularity because of its potential in cognitive and AI applications. NLU goes beyond the structural understanding of language to interpret intent, resolve context and word ambiguity, and even generate well-formed human language on its own. 