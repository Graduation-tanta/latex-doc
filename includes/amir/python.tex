\section{Python (programming language)}
\label{chap:Python (programming language)}

  %%%%%%%%%%%%%%%%%%%%%%%%%%%%
  % SUBSECTION               %
  %%%%%%%%%%%%%%%%%%%%%%%%%%%%
\subsection{Definition}  
Python is an interpreted high-level programming language for general-purpose programming. Created by Guido van Rossum and first released in 1991, Python has a design philosophy that emphasizes code readability, notably using significant whitespace. It provides constructs that enable clear programming on both small and large scales.\\
Python features a dynamic type system and automatic memory management. It supports multiple programming paradigms, including object-oriented, imperative, functional and procedural, and has a large and comprehensive standard library.\\
Python interpreters are available for many operating systems. CPython, the reference implementation of Python, is open source software[29] and has a community-based development model, as do nearly all of its variant implementations. CPython is managed by the non-profit Python Software Foundation.\cite{web006}
\subsection{Indentation}
Python is meant to be an easily readable language. Its formatting is visually uncluttered, and it often uses English keywords where other languages use punctuation. Unlike many other languages, it does not use curly brackets to delimit blocks, and semicolons after statements are optional.\\
Python uses whitespace indentation, rather than curly brackets or keywords, to delimit blocks. An increase in indentation comes after certain statements; a decrease in indentation signifies the end of the current block. Thus, the program's visual structure accurately represents the program's semantic structure.
\subsection{Libraries}
Python's large standard library, commonly cited as one of its greatest strengths, provides tools suited to many tasks in Scientific computing, Text processing, Image processing, Networking, Multimedia, Databases, and many more.\\
To mention some libraries like NumPy that provides some advance math functionalities, matplotlib numerical plotting library that is very useful for any data scientist or any data analyzer, nltk. Natural Language Toolkit to manipulate strings, Pandas for data manipulation and visualization, TensorFlow that enables quick training of artificial neural networks on large datasets, Keras also for building Neural Networks at a high-level of the interface based on Tensorflow and many more.
\subsection{Pros and Cons}
Advantages like :\cite{web007}\\
\indent$\bullet$\hspace{5pt}Its benefits in machine learning and deep learning are known.\\
\indent$\bullet$\hspace{5pt}Python is easy to learn for even a novice developer. Its code is easy to read and you can do a lot of things just by looking at it. Also, you can execute a lot of complex functionalities with ease, thanks to the standard library.\\
\indent$\bullet$\hspace{5pt}Supports multiple systems and platforms and Object Oriented Programming-driven.\\
\indent$\bullet$\hspace{5pt}With the introduction of Raspberry Pi, a card sized microcomputer, Python has expanded its reach to unprecedented heights. Developers can now build cameras, radios and games with ease. So, learning Python could open new avenues for you to create some out-of-the box gadgets.\\
\indent$\bullet$\hspace{5pt}large number of resources are available for Python and Allows to scale even the most complex applications with ease.\\
\indent$\bullet$\hspace{5pt}IT giants like Yahoo, Google, IBM, NASA, Nokia and Disney prefer Python.\\
Disadvantages :\\
\indent$\bullet$\hspace{5pt}Python is slower than C/C++ as it is a high-level language.\\
\indent$\bullet$\hspace{5pt}Python is not a very good language for mobile development .\\
\indent$\bullet$\hspace{5pt}Python is not a good choice for memory intensive tasks.\\
\indent$\bullet$\hspace{5pt}Python has limitations with database access . As compared to the popular technologies like JDBC and ODBC.\\
\indent$\bullet$\hspace{5pt}It faces high difficulties in building a high-graphic 3D game using Python.\\
\indent$\bullet$\hspace{5pt}Also its runtime errors and obstacles in multi-processor/multi-core work.
