\chapter{Introduction}
\label{chap:chap1}

\section{Section One}

  \subsection{Introduction}
  


Chatbots are computer programs that interact with users using natural languages. This technology started in the 1960�s; the aim was to see if chatbot systems could fool users that they were real humans, thereby passing the Turing test. However, chatbot systems are not only built to mimic human conversation and entertain users. Chatbots are typically used in dialog systems for various practical purposes. including customer service or information acquisition. Some chatterbots like ours use sophisticated natural language processing systems,
The term "ChatterBot" was originally coined by Michael Mauldin (creator of the first Verbot, Julia) in 1994 to describe these conversational programs. Today, most chatbots are either accessed via virtual assistants such as Google Assistant and Amazon Alexa, via messaging apps such as Facebook Messenger or WeChat, or via individual organizations' apps and websites. And like everything else, such technology evolved with time to become one of the most important technologies nowadays. Since we are in the age of semantic web They are now being used in almost everywhere. Chatbots can be classified into usage categories such as conversational commerce (e-commerce via chat), analytics, communication, customer support, design, education, entertainment, finance, food, games, health, marketing, news, shopping, social, sports, and travel.
For example: Instead of browsing the Amazon website, you might have a conversation with Alexa to find and order the product you need. Alexa is a voice-controlled intelligent personal assistant service that is part of the Amazon Echo smart speaker. Alexa might converse with you in the same way you might with a shop assistant in a retail store. It could provide personalized recommendations and assist with on-the-spot purchasing. Alexa could also automatically connect to your existing customer data and confirm your identity.
Chatbots May Appear to Replace Applications To the end user, it may appear that systems such as Siri and Alexa have replaced software applications (apps) when we ask them to book meetings in our calendars and purchase train tickets for us. In reality, they will encapsulate the apps and sit on top of them. While we may see fewer app icons on our smartphones, we will still need apps in the background for fulfillment and as data repositories.

\section{Section two}

  \subsection{User-Perspective}
From a user perspective it�s important to have things clear like question and answer crystal clear. 
As a user I want to ask a question on mind and get the perfect answer.
As a user the best way to ask is in a friendly chat.
As a user I want my questions to be handled correctly.
The chatbot design is the process that defines the interaction between the user and the chatbot. We will define the chatbot personality, and the overall interaction. It can be viewed as a subset of the conversational design. 
A conversation flow chart works well for a linear dialog. Using this chart, you can visualize a sequence of user-bot replicas even for a condition-based chatbot.
Meanwhile, dealing with advanced NLP-based chatbots implies working with a fluid dialog with many decision trees inside. In this case, it�s impossible to build a complete dialog tree because such chatbot�s behavior greatly depends on the context and user questions. Therefore, reasonable documentation for an NLP chatbot comprises of a list of intents, utterances, entities and actions in a table format.

\begin{figure}[H]%
    \center%
    \includegraphics[width=0.6\textwidth]{This PC/E/User-perspective.png}
     
    \caption[So in this, the user asks the question �How many of Warsaw�s inhabitants spoke polish in 1933?� 
The question is sent through the messenger bot to the server, then the document retriever retrieves the document that contains the answer from the context, then the answer �833,500� is read back to the user through the server.
]{building an inverted index}\label{fig:User Perspective}%
 \end{figure}
