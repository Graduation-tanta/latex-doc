
\chapter{Introduction}
\label{chap:chap1}

\section{Section One}

  \subsection{Introduction}
  


Chatbots are computer programs that interact with users using natural languages. This technology started in the 1960�s; the aim was to see if chatbot systems could fool users that they were real humans, thereby passing the Turing test. However, chatbot systems are not only built to mimic human conversation and entertain users. Chatbots are typically used in dialog systems for various practical purposes. including customer service or information acquisition. Some chatterbots like ours use sophisticated natural language processing systems,
The term "ChatterBot" was originally coined by Michael Mauldin (creator of the first Verbot, Julia) in 1994 to describe these conversational programs. Today, most chatbots are either accessed via virtual assistants such as Google Assistant and Amazon Alexa, via messaging apps such as Facebook Messenger or WeChat, or via individual organizations' apps and websites. And like everything else, such technology evolved with time to become one of the most important technologies nowadays. Since we are in the age of semantic web They are now being used in almost everywhere. Chatbots can be classified into usage categories such as conversational commerce (e-commerce via chat), analytics, communication, customer support, design, education, entertainment, finance, food, games, health, marketing, news, shopping, social, sports, and travel.
For example: Instead of browsing the Amazon website, you might have a conversation with Alexa to find and order the product you need. Alexa is a voice-controlled intelligent personal assistant service that is part of the Amazon Echo smart speaker. Alexa might converse with you in the same way you might with a shop assistant in a retail store. It could provide personalized recommendations and assist with on-the-spot purchasing. Alexa could also automatically connect to your existing customer data and confirm your identity.
Chatbots May Appear to Replace Applications To the end user, it may appear that systems such as Siri and Alexa have replaced software applications (apps) when we ask them to book meetings in our calendars and purchase train tickets for us. In reality, they will encapsulate the apps and sit on top of them. While we may see fewer app icons on our smartphones, we will still need apps in the background for fulfillment and as data repositories.

\section{Section two}

  \subsection{User-Perspective}
From a user perspective it�s important to have things clear like question and answer crystal clear. 
As a user I want to ask a question on mind and get the perfect answer.
As a user the best way to ask is in a friendly chat.
As a user I want my questions to be handled correctly.
The chatbot design is the process that defines the interaction between the user and the chatbot. We will define the chatbot personality, and the overall interaction. It can be viewed as a subset of the conversational design. 
A conversation flow chart works well for a linear dialog. Using this chart, you can visualize a sequence of user-bot replicas even for a condition-based chatbot.
Meanwhile, dealing with advanced NLP-based chatbots implies working with a fluid dialog with many decision trees inside. In this case, it�s impossible to build a complete dialog tree because such chatbot�s behavior greatly depends on the context and user questions. Therefore, reasonable documentation for an NLP chatbot comprises of a list of intents, utterances, entities and actions in a table format.

\begin{figure}[H]%
    \center%
    \includegraphics[width=0.6\textwidth]{This PC/E/User-perspective.png}
     
    \caption[So in this, the user asks the question �How many of Warsaw�s inhabitants spoke polish in 1933?� 
The question is sent through the messenger bot to the server, then the document retriever retrieves the document that contains the answer from the context, then the answer �833,500� is read back to the user through the server.
]{building an inverted index}\label{fig:User Perspective}%
 \end{figure}

\chapter{?Artificial Intelligence Primer}
\label{chap:chap2}

\section{Section One}

  \subsection{AI Introduction}
There are two types of chatbots
1. One operates based on a set of rules. It can only respond to very specific commands. If you, as the user don't use the right command or words, the chatbot doesn�t know what you mean like chatbots created using AIML.
2. The other type uses machine learning and artificial intelligence to provide the best response. We'll call these AI-powered chatbots. 
It understands language, as well as commands.
It has the ability to constantly learn from user interactions to become better at predicting their needs. 
It can chat in a similar way a staff member would with a person. 
It can store and categorize the information it receives from each interaction. 
It can assess information to identify which information is of no value and which isn't. 
It Knows where to store that information, so it can access it again in the future.
So as we can see AI is very important when it comes to chatbots, because it makes the chatbot very human like or even more by giving it the ability to do important tasks that includes the tasks mentioned in the type 2 of the chatbots.
AI-powered chatbots can interpret and carry out more complex requests - ones that involve a series of complex tasks. This means users don�t have to use specific commands to use it.

\section{Section Two}

  \subsection{Machine Learning}
Machine Learning is an idea to learn from examples and experience, without being explicitly programmed. Instead of writing code, you feed data to the generic algorithm, and it builds logic based on the data given.
�A computer program is said to learn from experience E with some class of tasks T and performance measure P if its performance at tasks in T, as measured by P, improves with experience E.� -Tom M. Mitchell
Machine Learning is a field which is raised out of Artificial Intelligence(AI). Applying AI, we wanted to build better and intelligent machines. But except for few mere tasks such as finding the shortest path between point A and B, we were unable to program more complex and constantly evolving challenges.There was a realisation that the only way to be able to achieve this task was to let machine learn from itself. This sounds similar to a child learning from its self. So machine learning was developed as a new capability for computers. And now machine learning is present in so many segments of technology, that we don�t even realise it while using it.
Finding patterns in data on planet earth is possible only for human brains. The data being very massive, the time taken to compute is increased, and this is where Machine Learning comes into action, to help people with large data in minimum time.
If big data and cloud computing are gaining importance for their contributions, machine learning as technology helps analyze those big chunks of data, easing the task of data scientists in an automated process and gaining equal importance and recognition.
The techniques we use for data mining have been around for many years, but they were not effective as they did not have the competitive power to run the algorithms. 
If you run deep learning with access to better data, the output we get will lead to dramatic breakthroughs which is machine learning
There are three kinds of Machine Learning Algorithms.
a. Supervised Learning
b. Unsupervised Learning
c. Reinforcement Learning
Supervised Learning
A majority of practical machine learning uses supervised learning.
In supervised learning, the system tries to learn from the previous examples that are given. (On the other hand, in unsupervised learning, the system attempts to find the patterns directly from the example given.)
Speaking mathematically, supervised learning is where you have both input variables (x) and output variables(Y) and can use an algorithm to derive the mapping function from the input to the output.
The mapping function is expressed as Y = f(X).
Unsupervised Learning
In unsupervised learning, the algorithms are left to themselves to discover interesting structures in the data.
Mathematically, unsupervised learning is when you only have input data (X) and no corresponding output variables.
This is called unsupervised learning because unlike supervised learning above, there are no given correct answers and the machine itself finds the answers.
Unsupervised learning problems can be further divided into association and clustering problems.
Association: An association rule learning problem is where you want to discover rules that describe large portions of your data, such as �people that buy X also tend to buy Y�.
Clustering: A clustering problem is where you want to discover the inherent groupings in the data, such as grouping customers by purchasing behaviour.
Reinforcement Learning
A computer program will interact with a dynamic environment in which it must perform a particular goal (such as playing a game with an opponent or driving a car). The program is provided feedback in terms of rewards and punishments as it navigates its problem space.
Using this algorithm, the machine is trained to make specific decisions. It works this way: the machine is exposed to an environment where it continuously trains itself using trial and error method.

\section{Section Three}

  \subsection{Neural Networks}
What Is A Neural Network?
The simplest definition of a neural network, more properly referred to as an 'artificial' neural network (ANN), is provided by the inventor of one of the first neurocomputers, Dr. Robert Hecht-Nielsen. He defines a neural network as:
"...a computing system made up of a number of simple, highly interconnected processing elements, which process information by their dynamic state response to external inputs.
In "Neural Network Primer: Part I" by Maureen Caudill, AI Expert, Feb. 1989
ANNs are processing devices (algorithms or actual hardware) that are loosely modeled after the neuronal structure of the mamalian cerebral cortex but on much smaller scales. A large ANN might have hundreds or thousands of processor units, whereas a mamalian brain has billions of neurons with a corresponding increase in magnitude of their overall interaction and emergent behavior. Although ANN researchers are generally not concerned with whether their networks accurately resemble biological systems, some have. For example, researchers have accurately simulated the function of the retina and modeled the eye rather well.
To better understand artificial neural computing it is important to know first how a conventional 'serial' computer and it's software process information. A serial computer has a central processor that can address an array of memory locations where data and instructions are stored. Computations are made by the processor reading an instruction as well as any data the instruction requires from memory addresses, the instruction is then executed and the results are saved in a specified memory location as required. In a serial system (and a standard parallel one as well) the computational steps are deterministic, sequential and logical, and the state of a given variable can be tracked from one operation to another.
In comparison, ANNs are not sequential or necessarily deterministic. There are no complex central processors, rather there are many simple ones which generally do nothing more than take the weighted sum of their inputs from other processors. ANNs do not execute programed instructions; they respond in parallel (either simulated or actual) to the pattern of inputs presented to it. There are also no separate memory addresses for storing data. Instead, information is contained in the overall activation 'state' of the network. 'Knowledge' is thus represented by the network itself, which is quite literally more than the sum of its individual components.
There are 6 types of Artificial Neural Networks Currently Being Used in Machine Learning
�Feedforward Neural Network � Artificial Neuron.
�Radial basis function Neural Network.
�Kohonen Self Organizing Neural Network.
�Recurrent Neural Network(RNN) � Long Short Term Memory.
�Convolutional Neural Network.
�Modular Neural Network.
\subsection{Deep Learning}
Deep learning is a specific subset of Machine Learning, which is a specific subset of Artificial Intelligence. For individual definitions:
�Artificial Intelligence is the broad mandate of creating machines that can think intelligently
�Machine Learning is one way of doing that, by using algorithms to glean insights from data (see our gentle introduction here)
�Deep Learning is one way of doing that, using a specific algorithm called a Neural Network
Deep Learning is important for one reason, and one reason only: we�ve been able to achieve meaningful, useful accuracy on tasks that matter. Machine Learning has been used for classification on images and text for decades, but it struggled to cross the threshold � there�s a baseline accuracy that algorithms need to have to work in business settings. Deep Learning is finally enabling us to cross that line in places we weren�t able to before.
Facebook has had great success with identifying faces in photographs by using Deep Learning. It�s not just a marginal improvement, but a game changer: �Asked whether two unfamiliar photos of faces show the same person, a human being will get it right 97.53 percent of the time. New software developed by researchers at Facebook can score 97.25 percent on the same challenge, regardless of variations in lighting or whether the person in the picture is directly facing the camera.�
Deep Learning is important because it finally makes these tasks accessible � it brings previously irrelevant workloads into the purview of Machine Learning. We�re just at the cusp of developers and business leaders understanding how they can use Machine Learning to drive business outcomes, and having more available tasks at your fingertips because of Deep Learning is going to transform the economy for decades to come.



\chapter{Technologies And Tools}
\label{chap:chap5}

\section{Section four}

  \subsection{Facebook Messenger API}
Many companies' chatbots run on messaging apps like Facebook Messenger (since 2016), WeChat (since 2013), WhatsApp, Kik, Slack, Line, Telegram, or simply via SMS. They are used for B2C customer service, sales and marketing. 
In 2016, Facebook Messenger allowed developers to place chatbots on their platform. There were 30,000 bots created for Messenger in the first six months, rising to 100,000 by September 2017.
Since September 2017, this has also been as part of a pilot program on WhatsApp. Airlines KLM and Aerom�xico  airlines had previously launched customer services on the Facebook Messenger platform.
The bots usually appear as one of the user's contacts but can sometimes act as participants in a group chat.
Many banks and insurers, media and e-commerce companies, airlines and hotel chains, retailers, health care providers, government entities and restaurant chains have used chatbots to answer simple questions, increase customer engagement, for promotion, and to offer additional ways to order from them. 
A 2017 study showed 4% of companies used chatbots. According to a 2016 study, 80% of businesses said they intended to have one by 2020. 

