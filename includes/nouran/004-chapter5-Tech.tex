\chapter{Technologies And Tools}
\label{chap:chap5}

\section{Section four}

  \subsection{Facebook Messenger API}
Many companies' chatbots run on messaging apps like Facebook Messenger (since 2016), WeChat (since 2013), WhatsApp, Kik, Slack, Line, Telegram, or simply via SMS. They are used for B2C customer service, sales and marketing. 
In 2016, Facebook Messenger allowed developers to place chatbots on their platform. There were 30,000 bots created for Messenger in the first six months, rising to 100,000 by September 2017.
Since September 2017, this has also been as part of a pilot program on WhatsApp. Airlines KLM and Aerom�xico  airlines had previously launched customer services on the Facebook Messenger platform.
The bots usually appear as one of the user's contacts but can sometimes act as participants in a group chat.
Many banks and insurers, media and e-commerce companies, airlines and hotel chains, retailers, health care providers, government entities and restaurant chains have used chatbots to answer simple questions, increase customer engagement, for promotion, and to offer additional ways to order from them. 
A 2017 study showed 4% of companies used chatbots. According to a 2016 study, 80% of businesses said they intended to have one by 2020. 

